\documentclass[journal,12pt,twocolumn]{IEEEtran}
%
\usepackage{setspace}
\usepackage{gensymb}
\usepackage{xcolor}
\usepackage{caption}
\usepackage{url}
%\usepackage{subcaption}
%\doublespacing
\singlespacing

%\usepackage{graphicx}
%\usepackage{amssymb}
%\usepackage{relsize}
\usepackage[cmex10]{amsmath}
\usepackage{mathtools}
%\usepackage{amsthm}
%\interdisplaylinepenalty=2500
%\savesymbol{iint}
%\usepackage{txfonts}
%\restoresymbol{TXF}{iint}
%\usepackage{wasysym}
\usepackage{amsthm}
\usepackage{mathrsfs}
\usepackage{txfonts}
\usepackage{stfloats}
\usepackage{cite}
\usepackage{cases}
\usepackage{subfig}
%\usepackage{xtab}
\usepackage{longtable}
\usepackage{multirow}
%\usepackage{algorithm}
%\usepackage{algpseudocode}
\usepackage{enumitem}
\usepackage{mathtools}

%\usepackage[framemethod=tikz]{mdframed}
\usepackage{listings}
\usepackage{listings}
    \usepackage[latin1]{inputenc}                                 %%
    \usepackage{color}                                            %%
    \usepackage{array}                                            %%
    \usepackage{longtable}                                        %%
    \usepackage{calc}                                             %%
    \usepackage{multirow}                                         %%
    \usepackage{hhline}                                           %%
    \usepackage{ifthen}                                           %%
  %optionally (for landscape tables embedded in another document): %%
    \usepackage{lscape}     


\usepackage{iithtlc}
\usepackage{tikz}
\usepackage{circuitikz}

%\usepackage{stmaryrd}


%\usepackage{wasysym}
%\newcounter{MYtempeqncnt}
\DeclareMathOperator*{\Res}{Res}
%\renewcommand{\baselinestretch}{2}
\renewcommand\thesection{\arabic{section}}
\renewcommand\thesubsection{\thesection.\arabic{subsection}}
\renewcommand\thesubsubsection{\thesubsection.\arabic{subsubsection}}

\renewcommand\thesectiondis{\arabic{section}}
\renewcommand\thesubsectiondis{\thesectiondis.\arabic{subsection}}
\renewcommand\thesubsubsectiondis{\thesubsectiondis.\arabic{subsubsection}}

% correct bad hyphenation here
\hyphenation{op-tical net-works semi-conduc-tor}

\lstset{
language=C,
frame=single, 
breaklines=true
}

%\lstset{
	%%basicstyle=\small\ttfamily\bfseries,
	%%numberstyle=\small\ttfamily,
	%language=Octave,
	%backgroundcolor=\color{white},
	%%frame=single,
	%%keywordstyle=\bfseries,
	%%breaklines=true,
	%%showstringspaces=false,
	%%xleftmargin=-10mm,
	%%aboveskip=-1mm,
	%%belowskip=0mm
%}

%\surroundwithmdframed[width=\columnwidth]{lstlisting}
\def\inputGnumericTable{}                                 %%
\lstset{
language=C,
frame=single, 
breaklines=true
}
 

\begin{document}
%

\theoremstyle{definition}
\newtheorem{theorem}{Theorem}[section]
\newtheorem{problem}{Problem}
\newtheorem{proposition}{Proposition}[section]
\newtheorem{lemma}{Lemma}[section]
\newtheorem{corollary}[theorem]{Corollary}
\newtheorem{example}{Example}[section]
\newtheorem{definition}{Definition}[section]
%\newtheorem{algorithm}{Algorithm}[section]
%\newtheorem{cor}{Corollary}
\newcommand{\BEQA}{\begin{eqnarray}}
\newcommand{\EEQA}{\end{eqnarray}}
\newcommand{\define}{\stackrel{\triangle}{=}}

\bibliographystyle{IEEEtran}
%\bibliographystyle{ieeetr}

\providecommand{\nCr}[2]{\,^{#1}C_{#2}} % nCr
\providecommand{\nPr}[2]{\,^{#1}P_{#2}} % nPr
\providecommand{\mbf}{\mathbf}
\providecommand{\pr}[1]{\ensuremath{\Pr\left(#1\right)}}
\providecommand{\qfunc}[1]{\ensuremath{Q\left(#1\right)}}
\providecommand{\sbrak}[1]{\ensuremath{{}\left[#1\right]}}
\providecommand{\lsbrak}[1]{\ensuremath{{}\left[#1\right.}}
\providecommand{\rsbrak}[1]{\ensuremath{{}\left.#1\right]}}
\providecommand{\brak}[1]{\ensuremath{\left(#1\right)}}
\providecommand{\lbrak}[1]{\ensuremath{\left(#1\right.}}
\providecommand{\rbrak}[1]{\ensuremath{\left.#1\right)}}
\providecommand{\cbrak}[1]{\ensuremath{\left\{#1\right\}}}
\providecommand{\lcbrak}[1]{\ensuremath{\left\{#1\right.}}
\providecommand{\rcbrak}[1]{\ensuremath{\left.#1\right\}}}
\theoremstyle{remark}
\newtheorem{rem}{Remark}
\newcommand{\sgn}{\mathop{\mathrm{sgn}}}
\providecommand{\abs}[1]{\left\vert#1\right\vert}
\providecommand{\res}[1]{\Res\displaylimits_{#1}} 
\providecommand{\norm}[1]{\lVert#1\rVert}
\providecommand{\mtx}[1]{\mathbf{#1}}
\providecommand{\mean}[1]{E\left[ #1 \right]}
\providecommand{\fourier}{\overset{\mathcal{F}}{ \rightleftharpoons}}
%\providecommand{\hilbert}{\overset{\mathcal{H}}{ \rightleftharpoons}}
\providecommand{\system}{\overset{\mathcal{H}}{ \longleftrightarrow}}
	%\newcommand{\solution}[2]{\textbf{Solution:}{#1}}
\newcommand{\solution}{\noindent \textbf{Solution: }}
\providecommand{\dec}[2]{\ensuremath{\overset{#1}{\underset{#2}{\gtrless}}}}
%\numberwithin{equation}{subsection}
\numberwithin{equation}{problem}
%\numberwithin{problem}{subsection}
%\numberwithin{definition}{subsection}
\makeatletter
\@addtoreset{figure}{problem}
\makeatother

\let\StandardTheFigure\thefigure
%\renewcommand{\thefigure}{\theproblem.\arabic{figure}}
\renewcommand{\thefigure}{\theproblem}


%\numberwithin{figure}{subsection}

%\numberwithin{equation}{subsection}
%\numberwithin{equation}{section}
%%\numberwithin{equation}{problem}
%%\numberwithin{problem}{subsection}
%\numberwithin{problem}{section}
%%\numberwithin{definition}{subsection}
%\makeatletter
%\@addtoreset{figure}{problem}
%\makeatother
%\makeatletter
%\@addtoreset{table}{problem}
%\makeatother

%\let\StandardTheFigure\thefigure
%\let\StandardTheTable\thetable
%%\renewcommand{\thefigure}{\theproblem.\arabic{figure}}
%\renewcommand{\thefigure}{\theproblem}
%\renewcommand{\thetable}{\theproblem}
%%\numberwithin{figure}{section}

%%\numberwithin{figure}{subsection}



\def\putbox#1#2#3{\makebox[0in][l]{\makebox[#1][l]{}\raisebox{\baselineskip}[0in][0in]{\raisebox{#2}[0in][0in]{#3}}}}
     \def\rightbox#1{\makebox[0in][r]{#1}}
     \def\centbox#1{\makebox[0in]{#1}}
     \def\topbox#1{\raisebox{-\baselineskip}[0in][0in]{#1}}
     \def\midbox#1{\raisebox{-0.5\baselineskip}[0in][0in]{#1}}

\vspace{3cm}

\title{ 
	\logo{
Mixing C and Assembly with Arduino
	}
}


% paper title
% can use linebreaks \\ within to get better formatting as desired
%\title{Matrix Analysis through Octave}
%
%
% author names and IEEE memberships
% note positions of commas and nonbreaking spaces ( ~ ) LaTeX will not break
% a structure at a ~ so this keeps an author's name from being broken across
% two lines.
% use \thanks{} to gain access to the first footnote area
% a separate \thanks must be used for each paragraph as LaTeX2e's \thanks
% was not built to handle multiple paragraphs
%

\author{Neelmani Gautam and G V V Sharma$^{*}$ %<-this  stops a space
\thanks{Neelmani is an undergraduate student at IIT Bhilai. email:neelmanig@iitbhilai.ac.in.  *The author is with the Department
of Electrical Engineering, Indian Institute of Technology, Hyderabad
502285 India e-mail:  gadepall@iith.ac.in. All content in the manual released under GNU GPL.  Free to use for anything.}% <-this % stops a space
%\thanks{J. Doe and J. Doe are with Anonymous University.}% <-this % stops a space
%\thanks{Manuscript received April 19, 2005; revised January 11, 2007.}}
}
% note the % following the last \IEEEmembership and also \thanks - 
% these prevent an unwanted space from occurring between the last author name
% and the end of the author line. i.e., if you had this:
% 
% \author{....lastname \thanks{...} \thanks{...} }
%                     ^------------^------------^----Do not want these spaces!
%
% a space would be appended to the last name and could cause every name on that
% line to be shifted left slightly. This is one of those "LaTeX things". For
% instance, "\textbf{A} \textbf{B}" will typeset as "A B" not "AB". To get
% "AB" then you have to do: "\textbf{A}\textbf{B}"
% \thanks is no different in this regard, so shield the last } of each \thanks
% that ends a line with a % and do not let a space in before the next \thanks.
% Spaces after \IEEEmembership other than the last one are OK (and needed) as
% you are supposed to have spaces between the names. For what it is worth,
% this is a minor point as most people would not even notice if the said evil
% space somehow managed to creep in.



% The paper headers
%\markboth{Journal of \LaTeX\ Class Files,~Vol.~6, No.~1, January~2007}%
%{Shell \MakeLowercase{\textit{et al.}}: Bare Demo of IEEEtran.cls for Journals}
% The only time the second header will appear is for the odd numbered pages
% after the title page when using the twoside option.
% 
% *** Note that you probably will NOT want to include the author's ***
% *** name in the headers of peer review papers.                   ***
% You can use \ifCLASSOPTIONpeerreview for conditional compilation here if
% you desire.




% If you want to put a publisher's ID mark on the page you can do it like
% this:
%\IEEEpubid{0000--0000/00\$00.00~\copyright~2007 IEEE}
% Remember, if you use this you must call \IEEEpubidadjcol in the second
% column for its text to clear the IEEEpubid mark.



% make the title area
\maketitle

%\newpage

%\tableofcontents


% IEEEtran.cls defaults to using nonbold math in the Abstract.
% This preserves the distinction between vectors and scalars. However,
% if the journal you are submitting to favors bold math in the abstract,
% then you can use LaTeX's standard command \boldmath at the very start
% of the abstract to achieve this. Many IEEE journals frown on math
% in the abstract anyway.

% Note that keywords are not normally used for peerreview papers.
%\begin{IEEEkeywords}
%Cooperative diversity, decode and forward, piecewise linear
%\end{IEEEkeywords}



% For peer review papers, you can put extra information on the cover
% page as needed:
% \ifCLASSOPTIONpeerreview
% \begin{center} \bfseries EDICS Category: 3-BBND \end{center}
% \fi
%
% For peerreview papers, this IEEEtran command inserts a page break and
% creates the second title. It will be ignored for other modes.
\IEEEpeerreviewmaketitle


%\documentclass{article}
%\usepackage[utf8]{inputenc}
%\usepackage{listings}
%\usepackage{graphicx}

%\usepackage{color}
%\definecolor{codegreen}{rgb}{0,0.6,0}
%\definecolor{codegray}{rgb}{0.5,0.5,0.5}
%\definecolor{codepurple}{rgb}{0.58,0,0.82}
%\definecolor{backcolour}{rgb}{0.95,0.95,0.92}
%\lstdefinestyle{mystyle}{
    %backgroundcolor=\color{backcolour},   
    %commentstyle=\color{codegreen},
    %keywordstyle=\color{magenta},
    %numberstyle=\tiny\color{codegray},
    %stringstyle=\color{codepurple},
    %basicstyle=\footnotesize,
    %breakatwhitespace=false,         
    %breaklines=true,                 
    %captionpos=b,                    
    %keepspaces=true,                 
    %numbers=left,                    
    %numbersep=5pt,                  
    %showspaces=false,                
    %showstringspaces=false,
    %showtabs=false,                  
    %tabsize=2
%}
 
%\lstset{style=mystyle}


%\title{Analog Design Through Arduino}
%\author{G V V Sharma* }

%\begin{document}

%\maketitle
\begin{abstract}
%%\boldmath
This manual shows how write a function in assembly and call it in a C program while programming the ATMega328P microcontroller in the Arduino.  This is done by controlling an LED. 
%
\end{abstract}
\section{Components}
\input{./figs/components.tex}
%\section{Measuring the resistance}
%\begin{problem}
%Connect the 5V pin of the Arduino to an extreme pin of the Breadboard shown in Fig. \ref{fig:breadboard}.  Let this pin be $V_{cc}$.
%\end{problem}
%%
%\begin{problem}
%Connect the GND pin of the Arduino to the opposite extreme pin of the Breadboard.
%\end{problem}
%%
%%
%\begin{problem}
%Let $R_1$ be the known resistor and $R_2$ be the unknown resistor.  Connect $R_1$ and $R_2$ in series such that $R_1$ is connected
%to GND and $R_2$ is connected to $V_{cc}$. Refer to Fig. \ref{fig:voltage_divider}
%\end{problem}
%%
%%
%\begin{figure}
%\centering
%%\includegraphics[width=\columnwidth]{./figs/voltage_divider.eps}
%\resizebox {\columnwidth} {!} {
%%\input{./figs/collpits.tex}
%\input{./figs/vrr.tex}
%}
%%\input{./figs/vrr.tex}
%%\includegraphics[width=\columnwidth]{./figs/voltage_divider.eps}
%\caption{Voltage Divider}
%\label{fig:voltage_divider}
%\end{figure}
%%
%
%\begin{problem}
%Connect the junction between the two resistors to  the A0 pin on the Arduino.
%\end{problem}
%%
%\begin{problem}
%Connect the arduino to the computer so that it is powered.
%\end{problem}
%%
%\begin{problem}
%Open the Arduino IDE and type the following code.  Open the {\em serial monitor} to view the output.
%\end{problem}
%%
%\lstinputlisting{./codes/resistance/resistance.ino}
\section{LED control}
\begin{problem}
Connect \textbf{pin 13}  of the Arduino to an LED through the resistor.
\end{problem}
\begin{problem}
Download the \textbf{Makefile} from 
\begin{lstlisting}
https://github.com/gadepall/EE2110/blob/master/gcc_assembly/codes/blink/Makefile
\end{lstlisting}

\end{problem}
\begin{problem}
\label{prob:avrgcc}
Write a C program for turning an LED on/off using AVR-GCC.
\end{problem}
\solution Save the following code in a file called \textbf{onoffavr.c}.  
\lstinputlisting{./codes/blink/onoffavr.c}
\begin{problem}
Suitably modify the \textbf{Makefile} to run the above code.
\end{problem}
\solution
In the \textbf{Makefile}, make the following changes.
\begin{lstlisting}
TARGET = onoffavr
ASRC =   
\end{lstlisting}
%
\begin{problem}
Run \textbf{make} in the terminal to turn the LED on.
\end{problem}
\begin{problem}
Modify Problem \ref{prob:avrgcc} to turn the LED off.
\end{problem}

\subsection{GCC with Assembly}
\begin{problem}
Write the \textbf{init} function as an assembly routine that can be called in the C program.
\end{problem}
\solution Save the following code in a file called \textbf{initasm.S}.  It is important that I/O (e.g. PORTB) registers be typed in capital letters.
\lstinputlisting{./codes/blink/initasm.S}
%
\begin{problem}
\label{prob:gcc-asm}
Modify the C program in Problem. \ref{prob:avrgcc} to call the  \textbf{init()} function from  \textbf{initasm.S}. Save it as \textbf{onoff.c}.
\end{problem}
\solution
\lstinputlisting{./codes/blink/onoff.c}
\begin{problem}
Modify the \textbf{Makefile} for linking the C and assembly code above and execute the program.
\end{problem}
\solution
\begin{lstlisting}
TARGET = onoff
ASRC =   initasm.S
\end{lstlisting}
\section{Two Assembly Routines}
\begin{problem}
Modify the C program in Problem. \ref{prob:gcc-asm} to include a function for displaying the output.  Name this file as \textbf{onoff2.c}
\end{problem}
\solution
\lstinputlisting{./codes/blink/onoff2.c}
\begin{problem}
Write an assembly routine for displaying output through pin 13.
\end{problem}
\solution
\lstinputlisting{./codes/blink/displedasm.S}
\begin{problem}
Modify the \textbf{Makefile} for linking \textbf{onoff2.c, initasm.S} and  \textbf{displedasm.S} and execute the program.
\end{problem}
\solution
\begin{lstlisting}
TARGET = onoff
ASRC =   initasm.S displedasm.S
\end{lstlisting}
%
\begin{problem}
Explain how the \textbf{disp\_led(0)} function in \textbf{onoff2.c} is related to \textbf{Register R24} in \textbf{disp\_led} routine in \textbf{displedasm.S}.
\end{problem}
\solution The function argument 0 in \textbf{disp\_led(0)} is passed on to R24 in the assembly routine for further operations.  Also, the registers R18-R24 are available for storing more function arguments according to the Table \ref{table:param_pass}.  More details are avilable in official ATMEL AT1886 reference.
\input{./figs/param_pass.tex}
\section{Blink}
\begin{problem}
Modify your codes for blinking an LED using the following routine for the delay.
\end{problem}
\lstinputlisting{./codes/blink/delayasm.S}
%\section{Decade Counter on a Seven Segment Display}
%\begin{problem}
%Write an assembly routine for the pin configuration in Table \ref{fig:arduino_atmega_pin_map}
%\end{problem}


\end{document}


